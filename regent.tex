\documentclass{article}%[twocolumn]

\usepackage[margin=1in]{geometry}
%~ \usepackage[svgnames]{xcolor}
\usepackage{soul,color}
	%\so{letterspacing}
	%\caps{CAPITALS, Small Capitals}
	%\ul{underlining}
	%\st{overstriking}
	%\hl{highlighting}
	%\setulcolor{red}
	%\setstcolor{green}
	%\sethlcolor{lightgray}
\usepackage{setspace}
	\doublespacing{}
\usepackage{mathptmx}
\title{}
\author{Dr. Karen Tatum}

\begin{document}
\pagestyle{empty}
%\maketitle
\section*{Personal Faith Statement}

%~ I am very proud of my Christian character and faith, derived from the exemplary life of my grandmother, Mammaw, who ensured my weekly presence in Sunday School and Church services at the University Church of Christ in Shreveport, Louisiana where I was born and raised. 
As a child, the Bible stories were my favorite part of Sunday. % were .
I was especially drawn towards Jesus' parables. 
I understood the surface meaning---the obvious moral message was pointed out and described in great detail in Sunday school and Church---but I somehow knew there was more to these stories than I could understand at that point in my life. 
I thought that I would thoroughly comprehend God's Word as an adult because every time I asked \emph{Why this?} or \emph{Why that?}, the adults said \emph{Because\dots}, as if it were universally understood. % My Mammaw always told me "don't cast your pearls before swine," but I didn't understand what she meant because she had plenty of pearls, and my grandaddy kept pigs not swine.
%~ 
%I was too young to realize that ... there was more to the stories...  
%I was drawn to 
%~ these stories called to me because they spoke of survival and the strength and glory that comes from faith in God. 
%~ \hl{something about learning though parables, beyond the simple moral message perhaps?}
%~ 
%I always wanted to know what they meant beyond the simple moral message... . 
%~ I was drawn to these stories because I wanted to know what they meant. 
%~ I was inherently curious and perceptive. 
%~ I was too young to realize that it would take several years of advanced literary study and graduate school to realize that I had an inherent talent for learning and teaching through stories. 
%~ That's why I became and English professor. 
%~ As a young Graduate Teaching Assistant, I found my teaaching methods began as Socratic discussion. 

 
%At age 10, I was baptized into the Church of Christ. I publicly affirmed my belief in the Father, Son, and Holy ghost, Jesus as God’s only son, who was crucified on the Cross so that we may be  saved. My faith in Jesus’ and my Mammaw’s unconditional love was the saving grace of my otherwise bleak childhood.
 
By God's grace, my first college courses were English and philosophy. 
Here I realized my inherent love of literature, my reading and writing skills, and my thirst for knowledge. I pursued a PhD in English simply because I wanted to understand as much about human behavior as I could  
%I was driven by the life lessons and understanding of human behavior that I received in all of my courses--so much so that I pursued my Bachelors, Masters, and Doctorate in English literature simply because I wanted to know more, as much as I possibly could about the reasons for human behavior. Like Jesus did, I wanted to help my students discover meaning without telling them what to think. understand that I was trying to intersuch as Noah's Ark, Daniel in the Lion's Den, and Jesus turning water into wine. %, just to name a few.
I didn't know it, but I was using stories to understand my own life just as, as a child,  I had used Bible stories to try and  understand my life. I still wanted to know as much as I could, but not about what Jesus meant in his parables; rather, I wanted to know what all the great philosophers and writers had to say about life. 
My favorite works of British literature -- Charlotte Bront\"e's \emph{Jane Eyre} for example -- are ones that carry Christ's fundamental messages of unconditional love, forgiveness, sympathy, and humility. Throughout college, graduate school and my early career, I focused on secular goals and guidance. I wasn't looking to Christ for direction. I bounced from state to state, from adjunct teaching to contract teaching. It was just the worst of times, tempestuous. My cats and I crossed the Mississippi river twice, working adjunct teaching positions, searching for tenure. It was a turbulent, tempestuous, dark time in my life.I just prayed "foxhole prayers" to get this job I wanted or that promotion. If I got what I wanted, I forgot to thank God, but if I didn't get what I wanted, He got an earful.  %but, in retrospect, 

Only now when experience has taught me to really look, I see His hand and footsteps in every move I made. 



%\hl{add a quick line (or two) about \emph{not} having Him in your life} \ul{and then}\dots{}My faith was constantly tested as I continuously questioned God's reasons for all of this turmoil in my life. 
%Throughout all the fear and doubt, I prayed ``I can do all things through Christ who strengthens me,'' until I found the courage and determination that could only come through Christ. 


In the Fall of 2007, I took a position as an Assistant Professor of English  at Norfolk State University. 
I loved teaching my favorite literary works, but the students were definitely the bright spot of my days. I could not imagine some of the hellish experiences they had had here on earth, yet they still believed they would triumph personally and professionally. Some days, they had more dedication than I did because they kept their faith and they weren't afraid to admit it. 
% , students at all levels, but especially the English majors I mentored and developed close relationships with.
%My students amazed me when they told me of the \hl{unimaginable  \dots word choice?} trials they had overcome and their determination to succeed in college and life. 
%True, most of them came into college with poor reading and writing skills, which made it difficult to pursue college-level English without backtracking; this was tough and trying most of the time. 
Their determination kept me going. Their bright eyes reminded me of the hope I had as a child. Their faith reminded me of the faith I'd abandoned. I began to pray every morning for the Holy Spirit to speak through me to my students, to help me direct them to God's will for them. As I practiced counting my blessings at night, I discovered many small ways Christ had helped us and my gratitude grew. %In my t This time, though, I remembered to count my blessings and than God for them every night. I practiced teaching in parables as Jesus did, because I wanted to help my students discover meaning without me telling them what to think. understand that I was trying to intersuch as Noah's Ark, Daniel in the Lion's Den, and Jesus turning water into wine. %, just to name a few.
I continue to be amazed by and grateful for the miracles Christ has helped me achieve in my life. Sometimes minute by minute, sometimes day by day, I work to truly surrender my will over to the care of Christ's will for me through prayer and/or repeating mantras like "seek ye first the kingdom of God and his righteousness and all these things will be added unto you" (Matt?) I have faith that God is much better equipped to care for me than I am, so I try to let him. I also know that with such faith, my life is much less stressful and far more serene. I don't come down with respiratory infections as much, and I have much more patience.   
%~ chool and my early career, I focused on secular goals and guidance. I wasn't looking to Christ for direction. I bounced from state to state, from adjunct teaching to contract teaching. It was just the worst of times, tempestuous. My cats and I crossed the Mississippi river twice, working adjunct teaching positions, searching for tenure. It was a turbulent, tempestuous, dark time in my life.I just prayed "foxhole prayers" to get this job I wanted or that promotion. If I got what I wanted, I forgot to thank God, but if I didn't get what I wanted, He got an earful.  %but, in retrospect, 
%~ \section{Why I want to work at Regent}
%~ 
I would like to join the English faculty of Regent University simply because I would like to share my faith in order to inspire others-the gift my students gave me--without fear of censure.
%In almost 20 years of teaching college writing and literature, I’ve become more and more aware of the similarities between teaching and preaching and the relationship between Christ’s teaching methods and mine. Christ taught through parables; he did not tell his disciples how to interpret them, he guided them towards understanding. Growing up enamored of these stories, I unwittingly discovered myself teaching literature in the same manner.  I love working to help my students see God’s light in them, giving them hope and faith in themselves for the first time in most of their lives. These 20 years have led me back to my own core faith through the privilege of teaching, showing, instructing students in the ways of life and the ways of life in Christ. I’ve seen the often blurred lines between preaching and teaching as I have become more blatant in sharing my Christian values with students as I see them expressed in the literary works under consideration. Simultaneously, my academic freedom to interpret literary works from a Christian perspective is limited in public state universities. Ironically, state universities ensure academic freedom, as long as it is non-denominational. 

%~ I’ve been teaching long enough to become tired of this hypocrisy and I am ready to teach in an environment that encourages me to fulfill my purpose in leading students to  their highest potential through Christ. 

\end{document}
