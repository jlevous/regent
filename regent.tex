\documentclass{article}

\usepackage[margin=1in]{geometry}
\usepackage{verse}
\usepackage{mathptmx}

\begin{document}
\pagestyle{empty}

\section*{Personal Faith Statement}

I am very proud of my Christian character and faith derived from the exemplary lives of both my grandparents who ensured my weekly presence in Sunday School and Church services of University church of Christ in Shreveport, Louisiana where I was born and raised.
As a child, my favorite part was the Bible stories like Noah's Ark, Daniel in the Lion's Den, and Jesus turning water into wine, just to name a few. 
I was too young to realize that I was drawn to these stories because they spoke of survival and the strength and glory that comes from a life lived in Christ.
At age 10, I was baptized into the Church of Christ. I repeated the Nicene Creed and affirmed my belief in the Father, Son, and Holy ghost, Jesus as God’s only son, who was crucified on the Cross so that we may be  saved. 
My faith in Jesus’ and my grandparents’ unconditional love was the saving grace of my otherwise bleak, abusive and traumatic childhood. 
Each taught me to hold on for ``the meek will inherit the earth'' and ``all will be revealed in time,'' as Mammaw said. Like Moses, Noah, and Christ's crucifixion, I held on to the belief that God had chosen me for His divine purpose, which wasn't mine to know at the time. 
My faith enabled me to endure a childhood of chronic eczema and chronic school bullying and domestic violence where Sundays with Mammaw were my only reprieve.

As soon as I turned 18, I left Shreveport for college in Houston, TX. 
I had turned my conscious mind from Christ, but He lived in my heart and soul patiently waiting for me to turn to him again which would not really happen for another decade. 
By God's grace, my first college courses were English and philosophy. 
Here I discovered my inherent love of and talents for reading and writing and my ravenous hunger for knowledge. 
For the first time in my life I received A's on my essays and exams, and I yearned for more of my professors' positive reinforcement--the only other I had received besides Mammaw and Christ. 
I stayed home studying every night and I eagerly attended every class period rather than skipping class to party with my friends as had become my high school custom. 
I was driven by the life lessons and understanding of human behavior that I received in all of my courses--history, psychology, world literature. I pursued my Bachelors, Masters, and Doctorate in English literature simply because I wanted to know more, as much as I possibly could about the reasons for human behavior. 
I didn’t know it, but I was using stories to understand my own life just as, as a child,  I had used Bible stories to understand my life. 
All of literature teaches some version of ethics and morality grounded in spiritual beliefs. 
My favorite works of British literature—the Canterbury Tales, Dr. Faustus, the Pilgrim’s Progress, the Victorian novels of Charlotte Bronte and Charles Dickens are no exception. 
I did not see god's hand in all of this, but looking back I do. 
I see god directing me through his will for me the whole time I had turned my back on him.
I see that now, but not then.
 
In the Fall of 1996, I left Houston to enter the doctoral program in English at the University of Alabama in the pretty small Southern town of Tuscaloosa, Alabama. 
In doing so, I ended a ten-year abusive relationship with my high school sweetheart. 
Once settled in Tuscaloosa, before my classes started, I began experiencing terrifying panic attacks although I didn't know what they were at the time. 
I didn't know until I went to the Emergency room thinking I was having a heart attack and dying. 
The ER doctor told me what they were and recommended a therapist. 
I began psychotherapy and the long, slow, painful process of coming to terms with my childhood in therapy and in my studies. 
This process lasted from 1996 to just recently and followed me through countless moves for teaching jobs, from Alabama to Virginia to Texas and back. 
This migrant lifestyle was tough on me and my cats, but through it I gained strength. 
My faith was constantly tested as I continuously questioned God's reasons for all of this turmoil in my life. 
I had turned back to God, but only to chide Him for not giving me what I wanted: stability and a tenure-track job as an English professor. 

Finally, in the Fall of 2007, I took a job as an English professor at Norfolk State University. 
I loved teaching my favorite literary works, but the best part of my job was the students, students at all levels, but especially the English majors I mentored and developed close relationships with. 
NSU students amazed me in the unimaginable trials they had overcome and their determination to succeed in college and life. 
True, most of them came into college with sixth-grade reading and writing skills, which made it difficult to pursue college-level English without backtracking; this was tough and trying most of the time. 
But their determination kept me going. 
Their Christian faith reminded me of mine. 
I turned back to Christ, not in a self-pitying was, but in a slowly thankful way.
 
In almost 20 years of teaching college writing and literature, I've become more and more aware of the similarities between teaching and preaching and the relationship between Christ's teaching methods and mine. 
To address the latter first, Christ taught through parables; he did not tell his disciples how to interpret them, he guided them towards understanding. 
Growing up enamored of these stories, I unwittingly discovered myself teaching literature in the same manner. 
My students are often taken aback with this method because they want me to tell them the meaning, but, as Mammaw always said to me, I say to them ``use the good brain the good Lord gave you.'' 
I love working to help them see God's light in them, giving them hope and faith in themselves for the first time in most of their lives. 
These 20 years have led me back to my own core faith through the privilege of teaching, showing, instructing students in the ways of life and the ways of life in Christ. 
I've seen the often blurred lines between preaching and teaching as I have become more blatant in sharing my Christian values with students as I see them expressed in the literary works under consideration. 
	
My grandparents are long gone, but I see myself practicing their examples frequently. 
I am now more aware of God's miracles in my life than I ever was before. 
Each day this grows deeper and deeper, richer and richer. 
Everyday I cite the Nicene Creed. 
I repeat it here in Affirmation of my faith in Christ.

\begin{verse}
We believe in one God, \\
the Father, the Almighty, \\
maker of heaven and earth, \\
of all that is, seen and unseen. \\>

We believe in one Lord, Jesus Christ, \\
the only Son of God, \\
eternally begotten of the Father, \\
God from God, Light from Light, \\
true God from true God, \\
begotten, not made, \\
of one Being with the Father. \\
Through him all things were made. \\
For us and for our salvation \\
he came down from heaven: \\
by the power of the Holy Spirit \\
he became incarnate from the Virgin Mary, \\
and was made man. \\
For our sake he was crucified under Pontius Pilate; \\
he suffered death and was buried. \\
On the third day he rose again \\
in accordance with the Scriptures; \\
he ascended into heaven \\
and is seated at the right hand of the Father. \\
He will come again in glory to judge the living and the dead, \\
and his kingdom will have no end. \\>

We believe in the Holy Spirit, the Lord, the giver of life, \\
who proceeds from the Father and the Son. \\
With the Father and the Son he is worshiped and glorified. \\
He has spoken through the Prophets. \\
We believe in one holy catholic and apostolic Church. \\
We acknowledge one baptism for the forgiveness of sins. \\
We look for the resurrection of the dead, \\
and the life of the world to come. Amen. \\

\hfill{---Episcopal Church Book of Common Prayer (1979), The Book of Common Prayer. New York: Church Publishing Incorporated. 2007. pp. 326-327. Retrieved 2013-02-18.}
\end{verse}


\end{document}
