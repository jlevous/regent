\documentclass{article}%[twocolumn]

\usepackage[margin=.5in]{geometry}
%~ \usepackage[svgnames]{xcolor}
\usepackage{soul,color}
	%\so{letterspacing}
	%\caps{CAPITALS, Small Capitals}
	%\ul{underlining}
	%\st{overstriking}
	%\hl{highlighting}
	%\setulcolor{red}
	%\setstcolor{green}
	%\sethlcolor{lightgray}
\usepackage{setspace}
	\onehalfspacing{}
\usepackage{mathptmx}
\title{}
\author{Dr. Karen Tatum}

\begin{document}
\pagestyle{empty}
%\maketitle
\section*{Personal Faith Statement}
%~ I am very proud of my Christian character and faith, derived from the exemplary life of my grandmother, Mammaw, who ensured my weekly presence in Sunday School and Church services at the University Church of Christ in Shreveport, Louisiana where I was born and raised. 
As a child, the Bible stories were my favorite part of Sunday. % were .
I was especially drawn towards Jesus' parables. 
I understood the surface meaning---the obvious moral message was pointed out and described in great detail in Sunday school and Church---but I somehow knew there was more to these stories than I could understand at that point in my life. 
I thought that I would thoroughly comprehend God's Word as an adult because every time I asked \emph{Why this?} or \emph{Why that?}, they said \emph{Because\dots}, as if it were universally understood. % My Mammaw always told me "don't cast your pearls before swine," but I didn't understand what she meant because she had plenty of pearls, and my grandaddy kept pigs not swine.
%~ 
%I was too young to realize that ... there was more to the stories...  
%I was drawn to 
%~ these stories called to me because they spoke of survival and the strength and glory that comes from faith in God. 
%~ \hl{something about learning though parables, beyond the simple moral message perhaps?}
%~ 
%I always wanted to know what they meant beyond the simple moral message... . 
%~ I was drawn to these stories because I wanted to know what they meant. 
%~ I was inherently curious and perceptive. 
%~ I was too young to realize that it would take several years of advanced literary study and graduate school to realize that I had an inherent talent for learning and teaching through stories. 
%~ That's why I became and English professor. 
%~ As a young Graduate Teaching Assistant, I found my teaaching methods began as Socratic discussion. 

 
%At age 10, I was baptized into the Church of Christ. I publicly affirmed my belief in the Father, Son, and Holy ghost, Jesus as God’s only son, who was crucified on the Cross so that we may be  saved. My faith in Jesus’ and my Mammaw’s unconditional love was the saving grace of my otherwise bleak childhood.

By God's grace, my first college courses were English and philosophy. 
Here I discovered my inherent love of and talents for reading and writing and my thirst for knowledge.  
%I was driven by the life lessons and understanding of human behavior that I received in all of my courses--so much so that I pursued my Bachelors, Masters, and Doctorate in English literature simply because I wanted to know more, as much as I possibly could about the reasons for human behavior. 
I didn't know it, but I was using stories to understand my own life just as, as a child,  I had used Bible stories to try and  understand my life. I still wanted to know as much as I could, but not about what Jesus meant in his parables; rather, I wanted to know what all the great philosophers and writers had to say about life. 
My favorite works of British literature -- Charlotte Bront\"e's \emph{Jane Eyre} for example -- are ones that carry Christ's fundamental messages of unconditional love, forgiveness, sympathy, and humility. 
Focused on secular goals and guidance as I was at the time, I wasn't looking to Christ for direction, but, in retrospect, I see His hand and footsteps in every move I made. 



\hl{add a quick line (or two) about \emph{not} having Him in your life} \ul{and then}\dots{}


My faith was constantly tested as I continuously questioned God's reasons for all of this turmoil in my life. 
Throughout all the fear and doubt, I prayed ``I can do all things through Christ who strengthens me,'' until I found the courage and determination that could only come through Christ. 





In the Fall of 2007, I took a posting as a Professor of English  at Norfolk State University. 
I loved teaching my favorite literary works, but the best part of my job was the students, students at all levels, but especially the English majors I mentored and developed close relationships with.
My students amazed me when they told me of the \hl{unimaginable  \dots word choice?} trials they had overcome and their determination to succeed in college and life. 
%True, most of them came into college with poor reading and writing skills, which made it difficult to pursue college-level English without backtracking; this was tough and trying most of the time. 
Their determination kept me going, \hl{their courage that \dots}, and their Christian faith reminded me of \hl{my own neglected faith?}. 
I began to pray every morning for the Holy Spirit to speak through me to my students \hl{-- their faith helped to renew my own}.  %Every evening, in recounting my day and counting my blessings, 



\hl{something remarkable.   Like Jesus did, I wanted to help my students discover meaning without telling them what to think.} % understand that I was trying to intersuch as Noah's Ark, Daniel in the Lion's Den, and Jesus turning water into wine. %, just to name a few.
I began ,and continue to be, amazed by and grateful for the miracles Christ has helped me achieve in my life. 

%~ \section{Why I want to work at Regent}
%~ 
%~ In almost 20 years of teaching college writing and literature, I’ve become more and more aware of the similarities between teaching and preaching and the relationship between Christ’s teaching methods and mine. Christ taught through parables; he did not tell his disciples how to interpret them, he guided them towards understanding. Growing up enamored of these stories, I unwittingly discovered myself teaching literature in the same manner.  I love working to help my students see God’s light in them, giving them hope and faith in themselves for the first time in most of their lives. These 20 years have led me back to my own core faith through the privilege of teaching, showing, instructing students in the ways of life and the ways of life in Christ. I’ve seen the often blurred lines between preaching and teaching as I have become more blatant in sharing my Christian values with students as I see them expressed in the literary works under consideration. Simultaneously, my academic freedom to interpret literary works from a Christian perspective is limited in public state universities. Ironically, state universities ensure academic freedom, as long as it is non-denominational. 

%~ I’ve been teaching long enough to become tired of this hypocrisy and I am ready to teach in an environment that encourages me to fulfill my purpose in leading students to  their highest potential through Christ. 

\end{document}
